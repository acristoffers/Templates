% !TeX root = document.tex
% !TeX encoding = UTF-8 Unicode

\IEEEtitleabstractindextext{%
\begin{abstract}
  O controle discreto no tempo se aproxima mais da realidade atual, onde o
  controle é feito de forma digital. As leis de controle implementadas
  utilizando sistemas discretos podem ser escritas com funções simples,
  normalmente de soma e multiplicação de termos, que são fáceis de interpretar
  e, principalmente, de implementar em um sistema digital. Por tratar o sistema
  como digital, ele leva em conta problemas da natureza desses sistemas, como o
  tempo de amostragem, normalmente apenas ignorados em implementações contínuas
  nesses sistemas. Este trabalho visa a implementação de dois controladores
  digitais em um sistema massa-mola em plano inclinado, visando demonstrar a
  síntese e implementação de tais controladores, e a análise desses sistemas.
\end{abstract}

\begin{IEEEkeywords}
  controle digital, jury, espaço de estados
\end{IEEEkeywords}
}
