% !TeX program = xelatex
% !TeX encoding = UTF-8 Unicode
% !BIB program = biber

% essa classe aceita as seguintes opções:
% abnd ou ieee pra estilo de citaçõ
% english ou french pra língua dos slides (não passar nada é português)
\documentclass[abnt]{slides}

\title{Template de Slides}
\subtitle{Exemplo de uso}

% Todo comando \add trabalha com uma lista e todo comando \set trabalha com um
% único elemento, ou seja, pode haver apenas um \setorientador mas vários
% \addauthor ou \addcoorientador. Todo comando \add e \set tem um \get
% correspondente (por exemplo, \getauthors, \getcoorientadores, \getorientador,
% \getmembrobanca). Esses comandos são definidos na classe slides.cls. No caso
% dos \set, os \get são implicitos, e não você não vai encontrá-los na classe.

\addauthor{Álan Crístoffer e Sousa}{acristoffers@gmail.com}
\setorientador{Prof.\ Dr.\ Valter Júnior de Souza Leite}
\addcoorientador{Prof.\ Dr.\ Ignacio Rubio Scola}

\setdepartamento{Engenharia Mecatrônica}
\setlocal{Divinópolis}
\setano{2020}

\preamble{}

% Essa classe usa o biblatex com backend biber ao invés do bibtex.
\addbibresource{bibliothek.bib}

% Controla o texto das captions das figuras. Neste caso, centraliza.
\captionsetup{justification=centering}

\begin{document}
\maketitle{}

% Use o ambiente slide ao invés do ambiente frame. Ele coloca a figura e
% nome dos autores no rodapé, além de centralizar verticalmente o conteúdo.
\begin{slide}{Sumário}
  % Para dividir um slide em colunas, use os ambientes columns e column.
  \begin{columns}[T] % T: top, c: center: B: bottom | alinhamento VERTICAL
    \begin{column}{0.48\textwidth} % coluna ocupando 48% da largura do texto
      \tableofcontents[sections={1-2}] % cria o índice apenas pras seções 1 a 2.
    \end{column}%
    \hfill%
    \begin{column}{0.48\textwidth}
      \tableofcontents[sections={3-}] % cria o índice da seção 3 até o final
    \end{column}%
  \end{columns}
\end{slide}

% Eu gosto de separar meus documentos \latex em vários arquivos, para manter
% organizado.
% !TeX root = document.tex
% !TeX encoding = UTF-8 Unicode

\section{Floats}
\subsection{Figuras}

\begin{slide}{Figura Única}
  \begin{figure}[ht!] % ht! é o posicionamento da figura. Inútil em slides.
    \centering % centraliza a figura no ambiente figure
    % height=0.3\textheight faz a imagem ter 30% da altura do texto. Outra
    % opção seria width=0.5\textwidth, por exemplo. Use a primeira pra
    % figuras formato retrato e a segunda pra figuras formato paisagem.
    % Ajuste o tamanho da figura usando a porcentagem.
    \includegraphics[height=0.3\textheight]{imgs/cefet}
    \caption{Logo do CEFET.} % texto que aparece abaixo da figura.
  \end{figure}
\end{slide}

\begin{slide}{Subfiguras}
  \begin{figure}[ht!]
    \centering
    \subfloat[One]{\includegraphics[height=0.3\textheight]{imgs/cefet}}\qquad
    \subfloat[Two]{\includegraphics[height=0.3\textheight]{imgs/cefet}}
    \caption{Duas logos do CEFET.}
  \end{figure}
\end{slide}

\subsection{Tabela}

\begin{slide}{Tabela}
  % Eu uso https://www.tablesgenerator.com pra gerar a tabela, e depois edito.
  \begin{table}
    \begin{tabular}{l|c|c|c|c} % alinhamento das células: Left/Center/Right
      \toprule
      Professor & Mecânica       & Eletrônica     & Controle       & Programação    \\ \midrule
      Lúcio     & \(\checkmark\) &                &                &                \\
      Valter    &                &                & \(\checkmark\) &                \\
      Thiago    &                &                &                & \(\checkmark\) \\
      Marlon    &                & \(\checkmark\) &                &                \\
      \bottomrule
    \end{tabular}
    \caption{Eixo de professores.}
  \end{table}
  % Este exemplo mostra como usar top/mid/bottom rules e linhas verticais. No
  % entanto, o povo do latex não recomenda usar isso dizendo que fica feio. No
  % fim do dia o documento é seu, use ou não se quiser.
\end{slide}

% !TeX root = document.tex
% !TeX encoding = UTF-8 Unicode

\section{Section 2}
\subsection{Subsection 2}

\begin{slide}{List}
    \vspace*{\fill}
    \begin{enumerate}
        \item One
        \item Two
        \item Three
    \end{enumerate}
    \vspace*{\fill}
\end{slide}

% !TeX root = document.tex
% !TeX encoding = UTF-8 Unicode

\section{Equações}
\subsection{Numeradas}

Equation:
\begin{equation}
  \dot{x} = Ax+Bu
\end{equation}

Align:
\begin{align}
  \dot{x} & = Ax+Bu \\
  y       & =Cx+Du
\end{align}

Equation+Aligned:
\begin{equation}
  \begin{aligned}
    \dot{x} & = Ax+Bu \\
    y       & =Cx+Du
  \end{aligned}
\end{equation}

\subsection{Não Numeradas}

Equation:
\begin{equation*}
  \dot{x} = Ax+Bu
\end{equation*}

Align:
\begin{align*}
  \dot{x} & = Ax+Bu \\
  y       & =Cx+Du
\end{align*}

Equation+Aligned:
\begin{equation*}
  \begin{aligned}
    \dot{x} & = Ax+Bu \\
    y       & =Cx+Du
  \end{aligned}
\end{equation*}

Align com epenas alguns items enumerados:
\begin{align}
  \dot{x} & = Ax+Bu           \\
  y       & = Cx+Du           \\
  c       & = Ex+Fu \nonumber % remove a numeração de uma equação
\end{align}


% A referência vai ser quebrada em quantos slides forem necessários.
\begin{frame}[allowframebreaks]{Referências}
  % Marca todas as referências do .bib como citadas, assim todas aparecem.
  \nocite{*}
  \printbibliography{}
\end{frame}
\end{document}
