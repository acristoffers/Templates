% !TeX root = document.tex
% !TeX encoding = UTF-8 Unicode

\section{Floats}
\subsection{Figuras}

\begin{slide}{Figura Única}
  \begin{figure}[ht!] % ht! é o posicionamento da figura. Inútil em slides.
    \centering % centraliza a figura no ambiente figure
    % height=0.3\textheight faz a imagem ter 30% da altura do texto. Outra
    % opção seria width=0.5\textwidth, por exemplo. Use a primeira pra
    % figuras formato retrato e a segunda pra figuras formato paisagem.
    % Ajuste o tamanho da figura usando a porcentagem.
    \includegraphics[height=0.3\textheight]{imgs/cefet}
    \caption{Logo do CEFET.} % texto que aparece abaixo da figura.
  \end{figure}
\end{slide}

\begin{slide}{Subfiguras}
  \begin{figure}[ht!]
    \centering
    \subfloat[One]{\includegraphics[height=0.3\textheight]{imgs/cefet}}\qquad
    \subfloat[Two]{\includegraphics[height=0.3\textheight]{imgs/cefet}}
    \caption{Duas logos do CEFET.}
  \end{figure}
\end{slide}

\subsection{Tabela}

\begin{slide}{Tabela}
  % Eu uso https://www.tablesgenerator.com pra gerar a tabela, e depois edito.
  \begin{table}
    \begin{tabular}{l|c|c|c|c} % alinhamento das células: Left/Center/Right
      \toprule
      Professor & Mecânica       & Eletrônica     & Controle       & Programação    \\ \midrule
      Lúcio     & \(\checkmark\) &                &                &                \\
      Valter    &                &                & \(\checkmark\) &                \\
      Thiago    &                &                &                & \(\checkmark\) \\
      Marlon    &                & \(\checkmark\) &                &                \\
      \bottomrule
    \end{tabular}
    \caption{Eixo de professores.}
  \end{table}
  % Este exemplo mostra como usar top/mid/bottom rules e linhas verticais. No
  % entanto, o povo do latex não recomenda usar isso dizendo que fica feio. No
  % fim do dia o documento é seu, use ou não se quiser.
\end{slide}
