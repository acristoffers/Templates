% !TeX root = document.tex
% !TeX encoding = UTF-8 Unicode

\chapter{Citações}
\section{Citações}

Para referenciar, há basicamente duas formas: \texttt{parencite} e
\texttt{textcite}. Eles são os equivalentes do biblatex aos comandos
\texttt{citep} e \texttt{citet} do bibtex. O primeiro serve para citar no fim do
parágrafo:

\begin{quote}
  Há quem diga que a lua é um queijo~\parencite{book:ogata}.
\end{quote}

O segundo para citar quando o nome do autor faz parte do texto:

\begin{quote}
  Segundo~\textcite{book:ogata}, a lua é um queijo.
\end{quote}
